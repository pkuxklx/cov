In the era of big data, we are gaining access to more and more auxiliary information apart from the observations of $\{\boldsymbol{X}_t\}_{t=1}^T$, which could potentially help us learn about the underlying structure of the covariance matrix (i.e., connectivity among entities). We believe this is the first paper that incorporates complex network information into the estimation of large covariance matrices. We believe a similar idea can be extended to many other settings, like the estimation of large dynamic covariance matrices. For example, dynamic network information could be well incorporated into the conditioning information set in \cite{chen2019new}.