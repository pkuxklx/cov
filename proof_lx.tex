\begin{lemma}
	For a series $\{ a_n \geq 0 \}$, denote $a_{(i)}$ as the $i$-th largest element. If $\sum_{i=k+1}^\infty a_{(i)} = O(k^{-\alpha})$, then $a_{(k)} = O(k^{-\alpha-1})$.
    \label{lemma_sum_single}
\end{lemma}

\begin{lemma}
	If $X_n = O_P(Y_n)$ and $Y_n = O_P(a_n)$, then $X_n = O_P(a_n)$.
	If $X_n = O_P(a_n)$ and $Y_n = O_P(b_n)$, then $X_n Y_n = O_P(a_n b_n)$.
    \label{lemma_OP}
\end{lemma}
\begin{proof}{of lemma \ref{lemma_OP}}

(1)
$\forall \epsilon, \exists M_1(\epsilon), N_1(\epsilon), \forall n > N_1, \Pr \{ |X_n/Y_n| > M_1 \} < \epsilon $. Similarly for $Y_n = O_P(a_n)$

We choose $M = \max\{ M_1^2(\frac{\epsilon}{2}), M_2^2(\frac{\epsilon}{2}) \}$, $N = \max \{ N_1(\frac{\epsilon}{2}), N_2(\frac{\epsilon}{2}) \}$.

Then $\forall n > N(\epsilon)$, we have
\begin{equation}
    \begin{split}
        \Pr \{ |\frac{X_n}{a_n}| > M \} &= \Pr\{ |\frac{X_n}{Y_n}| > \sqrt{M} \ \land \ |\frac{X_n}{a_n}| > M \} + \Pr \{ |\frac{X_n}{Y_n}| \leq \sqrt{M} \ \land \ |\frac{X_n}{a_n}| > M \} \\
        &\leq \Pr \{ |\frac{X_n}{Y_n}| > \sqrt{M} \} + \Pr \{ |\frac{Y_n}{a_n}| > \sqrt{M} \} \\
        &< \epsilon
    \end{split}
\end{equation}

(2)

$\forall \epsilon, \exists M_1(\epsilon), N_1(\epsilon), \forall n>N_1, \Pr \{ |X_n/a_n| > M_1 \} < \epsilon $. Similarly for $Y_n = O_P(b_n)$.

We choose $M = \max\{ M_1^2(\frac{\epsilon}{2}), M_2^2(\frac{\epsilon}{2}) \}$, $N = \max \{ N_1(\frac{\epsilon}{2}), N_2(\frac{\epsilon}{2}) \}$.

Then $\forall n>N(\epsilon)$, we have
\begin{equation}
	\begin{split}
		\Pr \{ |\frac{X_n Y_n}{a_n b_n}| > M \} &=  
			\Pr \{ |\frac{X_n}{a_n}| > \sqrt M \ \land\ |\frac{X_n Y_n}{a_n b_n}| > M  \} + 
			\Pr \{ |\frac{X_n}{a_n}| \leq \sqrt M \ \land\ |\frac{X_n Y_n}{a_n b_n}| > M  \} \\
		&\leq  \Pr \{ |\frac{X_n}{a_n}| > \sqrt M \} + 
			\Pr \{ |\frac{Y_n}{b_n}| > \sqrt M \} \\
		&< \epsilon.
	\end{split}
\end{equation}

\end{proof}

%\proof{ of theorem \ref{theorem2}}
\begin{proof}{of theorem \ref{theorem2}}

Denote $B_{R, k}(\hat R) = B_{C, k}(\hat R) = [\hat r_{ij} I(j\in S_{k}^{c_i} \land i\in S_{k}^{c_j})]$, and substituting $\hat r_{ij}$ with $r_{ij}$ yields $B_{C, k}(R)$. The notations are consistent with (\ref{network banding estimator}). We decompose as follows: 
\begin{equation}
	\begin{split}
        \lVert B_{\hat{C}, k}(\hat R) - R \rVert &\leq 
            \lVert B_{\hat{C}, k}(\hat R) - B_{\hat{C}, k}(R) \rVert + 
        	\lVert  B_{\hat{C}, k}(R) - B_{C, k}(R) \rVert + 
            \lVert B_{C, k}(R) - R \rVert \\
        &= \mathrm{I} + \mathrm{II} + \mathrm{III},
    \end{split} 
    \label{decompose}
\end{equation}
    
Bound term $\mathrm{I}$. 
\begin{equation}
	\begin{split}
	    \mathrm{I} &\leq \| B_{\hat{C}, k}(\hat R) - B_{\hat{C}, k}(R) \|_{(\infty, 
	        \infty)} \\
    	&= \max_i \Sigma_{j:\ j\in S_{k}^{\hat c_i} \land i\in S_{k}^{\hat c_j}} |\hat 
    	    r_{ij} - r_{ij}| \\
        &\leq (\max |S_{k}^{\hat c_i}|)  (\max_{i,j} 
            |\hat r_{ij} - r_{ij}|)  \\
        &= O_P(k \sqrt\frac{\log N}{T}). 
	\end{split}  
	\label{I res}
  \end{equation}
Bound term $\mathrm{III}$. 
\begin{equation}
    \begin{split}
	    \mathrm{III} &\leq \max_i \Sigma_{j:\ j \notin S_{k}^{c_i} \lor i 
	        \notin  S_{k}^{c_j}} |r_{ij}| \\
    	&\leq \max_i \Sigma_{j:\ j \notin S_{k}^{c_i}} |r_{ij}| + \max_i \Sigma_{j:\
    	    j\in S_{k}^{c_i} \land i\notin S_{k}^{c_j}} |r_{ij}|  \\
		&= \mathrm{III_1} + \mathrm{III_2}.
    \end{split}  
    \label{III}  
\end{equation}
By lemma \ref{lemma_sum_single}, we have 
\begin{equation}
	\begin{split}
		\mathrm{III_1}  &= O(  c_0(N)^{\frac{1}{q}} \Sigma_{i=k}^p 
		    i^{-\frac{\alpha+1}{q}}  ) \\
  	    &= O( c_0(N)^{\frac{1}{q}} k^{-\frac{\alpha+1}{q}+1} )
    \end{split}
\end{equation}
and 
\begin{equation}
	\begin{split}
		\mathrm{III_2} &\leq |S_{k}^{c_i}| ( \max_{i,j:\ i\notin S_{k}^{c_j}} |r_{ij}| )     \\
        &= k \ O( c_0(N)^{\frac{1}{q}} k^{-\frac{\alpha+1}{q}})  \\
        &= O( c_0(N)^{\frac{1}{q}} k^{-\frac{\alpha+1}{q}+1} ).
	\end{split}
\end{equation}
So we conclude 
\begin{equation}
    \mathrm{III} = O( c_0(N)^{\frac{1}{q}} k^{-\frac{\alpha+1}{q}+1} ). 
    \label{III res}
\end{equation}
    
Bound term $\mathrm{II}$. We denote 
\begin{equation}
	\begin{split}
		A_{k} &:= \{ (i,j) | j\in S_{k}^{c_i} \land i \in S_{k}^{c_j} \} \\
        \hat A_k &:= \{ (i,j) | j\in S_{k}^{\hat c_i} \land i \in S_{k}^{\hat c_j} \}, 
	\end{split}
\end{equation}
We further decompose $\mathrm{II}$ as: 
\begin{equation}
	\begin{split}
		\mathrm{II} &\leq \max_i \Sigma_{j:\ (i,j)\in A_k \triangle \hat A_k }       |r_{ij}| \\
        &\leq \max_i \Sigma_{j:\ (i,j) \in (A_k - \hat A_k)} |r_{ij}| + 
        	\max_i \Sigma_{j:\ (i,j) \in (\hat A_k - A_k)} |r_{ij}| \\
        &= \mathrm{IV} + \mathrm{V}.
	\end{split}
    \label{II decompose}
\end{equation}    
Because $(\hat A_k - A_k) \subseteq \overline{A_k}$, term $\mathrm{V} = O_P(c_0(N)^{\frac{1}{q}} k^{-\frac{\alpha+1}{q}+1})$ can be bounded in the same way as $\mathrm{III}$. To bound term $\mathrm{IV}$, notice that the equation in assumption \ref{asmp:framework2} indicates 
\begin{equation}
	\begin{split}
	    & \forall i, S_{\eta k}^{c_i} \subseteq S_{k}^{\hat c_i} \\
        \Rightarrow & \\
        & A_k - \hat A_k \subseteq \overline{ A_{\eta k} }  \\
        \Rightarrow & \\
		& \mathrm{IV} \leq \max_i \Sigma_{j:\ (i,j)\in\overline{A_{\eta k}}} |r_{ij}| := \mathrm{III'}  \\
        %\Rightarrow & \\
        % & \mathrm{IV} = O( \mathrm{III} ).
	\end{split}
 \label{impli:asmp:framework2}
\end{equation} 
Similarly as $\mathrm{III}$, it is easy to show that $\mathrm{III'} = O( c_0(N)^{\frac{1}{q}} k^{-\frac{\alpha+1}{q}+1} ) $. Note that (\ref{impli:asmp:framework2}) is just an implication but not a ground truth. 

$\forall \epsilon>0, \exists T_0$

Next, I will show that $\mathrm{IV} = O_P(\mathrm{III'})$ with assumption \ref{asmp:framework2}, and then together with lemma \ref{lemma_OP} we have
\begin{equation}
	\begin{split}
		\mathrm{IV} &= O_P( \mathrm{III} ) \\
        &= O_P(c_0(N)^{\frac{1}{q}} k^{-\frac{\alpha+1}{q}+1}) .
	\end{split} 
	\label{VI}
\end{equation}
To sum up, 
\begin{equation}
	\mathrm{II} = O_P( c_0(N)^{\frac{1}{q}}k^{-\frac{\alpha+1}{q}+1}  ). 
	\label{II res}
\end{equation}
By (\ref{I res}), (\ref{II res}), (\ref{III res}) and (\ref{k_NT}), we attain the rate in (\ref{theorem2_R_rate}).

Now consider the covariance estimation error. Similar to the argument in (\cite{liu2014EC2})'s Appendix E, we have 
\begin{equation}
	\begin{split}
		B_{\hat C, k}(\hat \Sigma) - \Sigma =& \hat D B_{\hat C, k}(\hat R) \hat D - DRD \\
        =& (\hat D - D) R D + (\hat D - D) (B_{\hat C, k}(\hat R) - R) D + (\hat D - D) (B_{\hat C, k}(\hat R) - R) (\hat D - D) + \\
    	& (\hat D - D) R (\hat D - D) + D (B_{\hat C, k}(\hat R) - R) D + D (B_{\hat C, k}(\hat R) - R) (\hat D - D) + \\
        & DR(\hat D - D)
	\end{split}
    \label{cov decompose}
\end{equation}
Then by $\|AB\|_2 \leq \|A\|_2 \|B\|_2$ we have 
\begin{equation}
	\begin{split}
		\lVert B_{\hat C, k}(\hat \Sigma) - \Sigma \rVert_2 \leq & 
		    \underbrace{ 2 \lVert \hat D -	D \rVert_2 \lVert R \rVert_2 \lVert D \rVert_2 }_{T_1} + 
		    \underbrace{ 2\lVert \hat D - D \rVert_2 \lVert B_{\hat C, k}(\hat R) - R \rVert_2 \lVert D \rVert_2 }_{T_2} \\
	    & + \underbrace{ \lVert \hat D - D \rVert_2^2 \lVert B_{\hat C, k}(\hat R) - R \rVert_2 }_{T_3} + 
	        \underbrace{ \lVert \hat D - D \rVert_2^2 \lVert R \rVert_2 }_{T_4} + 
	        \underbrace{ \lVert B_{\hat C, k}(\hat R) - R \rVert_2 \lVert D \rVert_2^2 }_{T_5} \\
        \leq & \underbrace{ 2 \sqrt{\kappa} \lambda_{\max } (R) \eta_1     }_{T_1} + 
            \underbrace{ 2\sqrt{\kappa} \eta_1 \eta_2}_{T_2} + 
            \underbrace{ \eta_1^2 \eta_2 }_{T_3} + 
            \underbrace{ \lambda_{\max}(R) \eta_1^2 }_{T_4} + 
            \underbrace{\kappa \eta_2}_{T_5} .
    \end{split}
    \label{cor2cov 2}
\end{equation}
where $\eta_1 =  \lVert \hat D - D \rVert_2 = O_P( \sqrt\frac{\log N}{T} )$ is given by assumption \ref{Gaussian asmp}, $\eta_2 = \lVert B_{\hat C, k}(\hat R) - R \rVert_2$ is already known, and $\lambda_{\max} (R) \leq 1/\varepsilon$. Because we assume $\eta_1$ and $\eta_2$ tend to $0$, we have  
\begin{equation}
	\begin{split}
		\lVert \hat B_{\hat C, k}(\hat \Sigma) - \Sigma \rVert_2 
        	= O( T_5 ) 
        = O_P (  
        	c_0(N)^\frac{1}{\alpha+1}  
            ( \frac{\log N}{T} )^\frac{\alpha-q+1}{2(\alpha+1)}  ) \\
	\end{split}	
\end{equation}
\end{proof}

\begin{proof}{of theorem \ref{theorem3}}

We can decompose in the same way as (\ref{decompose}). 
\begin{equation}
	\begin{split}
        \lVert B_{\hat{C}, k}(\hat R) - R \rVert_F^2 &\leq 
            3 (\lVert B_{\hat{C}, k}(\hat R) - B_{\hat{C}, k}(R) \rVert_F^2 + 
        	\lVert  B_{\hat{C}, k}(R) - B_{C, k}(R) \rVert_F^2 + 
            \lVert B_{C, k}(R) - R \rVert_F^2 ) \\
        &= 3 (\mathrm{I}^2 + \mathrm{II}^2 + \mathrm{III}^2).
    \end{split} 
    \label{decompose}
\end{equation}
Other arguments are also similar.
\begin{equation}
    \begin{split}
        \mathrm{I}^2 &= \lVert B_{\hat{C}, k}(\hat R) - B_{\hat{C}, k}(R) \rVert_F^2 \\   
        &\leq N \max_i \Sigma_{j:\ j\in S_{k}^{\hat c_i} \land i\in S_{k}^{\hat c_j}} 
            |\hat r_{ij} - r_{ij}|^2 \\
        &= O_P(Nk \frac{\log N}{T}).
    \end{split}
\end{equation}
\begin{equation}
    \begin{split}
        \mathrm{III}^2 &= \lVert B_{C,k}(R) - R \rVert_F^2 \\
        &\leq N \max_i 
            \Sigma_{j:\ j \notin S_{k}^{c_i} \lor i \notin S_{k}^{c_j}}  
            r_{ij}^2 \\
        &= O(N c_0(N)^{\frac{2}{q}} k^{-\frac{2(\alpha+1)}{q}+1} )
    \end{split}
\end{equation} 
\begin{equation}
    \begin{split}
        \mathrm{II}^2 &= \lVert B_{\hat{C}, k}(R) - B_{C, k}(R)  \rVert_F^2 \\
        &= \Sigma_{i,j:\ (i,j)\in A_k \triangle \hat A_k}  r_{ij}^2 \\
        &\leq N \max_i \Sigma_{j:\ (i,j)\in A_k \triangle \hat A_k}  r_{ij}^2 \\
        &= O_P(N c_0(N)^{\frac{2}{q}} k^{-\frac{2(\alpha+1)}{q}+1} )
    \end{split}
\end{equation}
Then we attain the rate in (\ref{theorem3_R_rate}). There is another trivial bound $\frac{1}{N} \lVert \hat B_{\hat C, k}(\hat R) - R \rVert_F^2 \leq \lVert \hat B_{\hat C, k}(\hat R) - R \rVert_2^2$, which is obviously worse than (\ref{theorem3_R_rate}). 

For the covariance estimation, we have 
\begin{equation}
	\begin{split}
		\lVert \hat B_{\hat C,k}(\hat \Sigma) - \Sigma \rVert_F \leq & 
		    \underbrace{ 2 \lVert \hat D -	D \rVert_F \lVert R \rVert_2 \lVert D \rVert_2 }_{\leq \sqrt N T_1} + 
		    \underbrace{ 2\lVert \hat D - D \rVert_2 \lVert \hat B_{\hat C,k}(\hat R) - R \rVert_F \lVert D \rVert_2 }_{T_6} \\
	    & + \underbrace{ \lVert \hat D - D \rVert_2^2 \lVert \hat B_{\hat C,k}(\hat R) - R \rVert_F }_{T_7} + 
	        \underbrace{ \lVert \hat D - D \rVert_2^2 \lVert R \rVert_F }_{\leq \sqrt N T_4} + 
	        \underbrace{ \lVert \hat B_{\hat C,k}(\hat R) - R \rVert_F \lVert D \rVert_2^2 }_{T_8} \\
        \leq & \underbrace{ 2 \sqrt N \sqrt \kappa \lambda_{\max}(R) \eta_1 }_{ \sqrt N T_1} + 
            \underbrace{ 2 \sqrt N \sqrt{\kappa} \eta_1 \eta_3}_{T_6} + 
            \underbrace{ \sqrt N \eta_1^2 \eta_3 }_{T_7} + 
            \underbrace{ \sqrt N \lambda_{\max}(R) \eta_1^2 }_{\sqrt N T_4} + 
            \underbrace{\sqrt N \kappa \eta_3}_{T_8} .
    \end{split}
    \label{cor2cov F}
\end{equation}
where $\eta_3 = \frac{1}{\sqrt N} \| B_{\hat C, k}(\hat R) - R \|_F $. Above we use two properties $\|A\|_F \leq \sqrt N \|A\|_2$ and $\|AB\|_F \leq \|A\|_2 \|B\|_F$. So we get (\ref{theorem3_S_rate}):
\begin{equation}
	\begin{split}
		\| \hat B_{\hat C, k}(\hat \Sigma) - \Sigma \|_F = O(T_8) 
        = O_P(  \sqrt N  c_0(N)^\frac{1}{2(\alpha + 1)} (\frac{\log N}{T})^\frac{2\alpha - q + 2}{4(\alpha + 1)}  ) 
	\end{split}
\end{equation}
\end{proof}
